% !TeX program = pdflatex
\documentclass[12pt,openright,oneside]{report}

% ========================= Codificación y tipografía (pdfLaTeX) =========================
\usepackage[utf8]{inputenc}
\usepackage[T1]{fontenc}
\usepackage{lmodern}

% ========================= Idiomas =========================
\usepackage[spanish,english]{babel}
\selectlanguage{spanish}

% ========================= Geometría, espaciado y tablas =========================
\usepackage{geometry}
\geometry{a4paper, margin=2.5cm}
\usepackage{setspace}
% \onehalfspacing % Activa si tu normativa lo exige
\usepackage{booktabs}
\usepackage{longtable}
\usepackage{multirow}
\usepackage{array}

% ========================= Gráficos y dibujo =========================
\usepackage{graphicx}
\graphicspath{{./}{figuras/}} % Ajusta si tus logos están en otra ruta
\usepackage{float}
\usepackage{subcaption}
\usepackage{tikz}
\usepackage{pgfplots}
\usepgfplotslibrary{colorbrewer}
\pgfplotsset{width=8cm,compat=1.18}

% ========================= Matemáticas y entornos =========================
\usepackage{amsmath, amssymb, amsthm, bm}
\usepackage{empheq}
\usepackage{mdframed}

% ========================= Colores =========================
\usepackage{xcolor}
\definecolor{ocre}{RGB}{243,102,25}
\definecolor{mygray}{RGB}{243,243,244}
\definecolor{deepGreen}{RGB}{26,111,0}
\definecolor{shallowGreen}{RGB}{235,255,255}
\definecolor{deepBlue}{RGB}{61,124,222}
\definecolor{shallowBlue}{RGB}{235,249,255}
\definecolor{ipnBrown}{RGB}{123,48,42}
\definecolor{gold}{RGB}{212,175,55}

% Caja sencilla
\newcommand\orangebox[1]{\fcolorbox{ocre}{mygray}{\hspace{1em}#1\hspace{1em}}}

% Entornos tipo teorema (nombres en inglés como en tu versión)
\newtheoremstyle{mytheoremstyle}{3pt}{3pt}{\normalfont}{0cm}{\rmfamily\bfseries}{}{1em}{{\color{black}\thmname{#1}~\thmnumber{#2}}\thmnote{\,--\,#3}}
\newtheoremstyle{myproblemstyle}{3pt}{3pt}{\normalfont}{0cm}{\rmfamily\bfseries}{}{1em}{{\color{black}\thmname{#1}~\thmnumber{#2}}\thmnote{\,--\,#3}}
\theoremstyle{mytheoremstyle}
\newmdtheoremenv[linewidth=1pt,backgroundcolor=shallowGreen,linecolor=deepGreen,leftmargin=0pt,innerleftmargin=20pt,innerrightmargin=20pt]{theorem}{Theorem}[section]
\theoremstyle{mytheoremstyle}
\newmdtheoremenv[linewidth=1pt,backgroundcolor=shallowBlue,linecolor=deepBlue,leftmargin=0pt,innerleftmargin=20pt,innerrightmargin=20pt]{definition}{Definition}[section]
\theoremstyle{myproblemstyle}
\newmdtheoremenv[linecolor=black,leftmargin=0pt,innerleftmargin=10pt,innerrightmargin=10pt]{problem}{Problem}[section]

% ========================= Encabezados y pies =========================
\usepackage{fancyhdr}
\pagestyle{fancy}
\fancyhf{}
\fancyhead[LE,RO]{\thepage}
\fancyhead[LO]{\nouppercase{\leftmark}}
\fancyhead[RE]{\nouppercase{\rightmark}}

% ========================= Hipervínculos y metadatos =========================
\usepackage[hidelinks]{hyperref}
\hypersetup{
  pdftitle={Prueba de Proyecto},
  pdfauthor={Luis Alberto Cordero Montes de Oca},
  pdfsubject={Arquitectura IoT para la detección de consumos anómalos},
  pdfkeywords={IoT, energía, anomalías, tiempo real}
}

% ========================= Glosarios/Acrónimos (opcional) =========================
\usepackage[acronym]{glossaries}
\makeglossaries
\newacronym{iot}{IoT}{Internet de las Cosas}
\newacronym{mqtt}{MQTT}{Message Queuing Telemetry Transport}

% ========================= Bibliografía (opcional, recomendado) =========================
% Descomenta si vas a usar .bib
\usepackage[backend=biber,style=ieee,sorting=nty]{biblatex}
\addbibresource{referencias.bib}

% ========================= Título y autor (para metadatos; no se usa portada automática) =========================
\title{Prueba de Proyecto}
\author{Luis Alberto Cordero Montes de Oca}

\begin{document}

% ========================= Portada (MANTENIDA TAL CUAL) =========================
\begin{titlepage}
    \thispagestyle{empty}
    \newgeometry{left=2.5cm,right=2.5cm,top=2.5cm,bottom=2.5cm}

    \begin{tikzpicture}[remember picture, overlay]
        % Logo IPN (arriba izquierda)
        \node[anchor=north west, xshift=1cm, yshift=-2cm] at (current page.north west) {
            \includegraphics[height=3cm]{logo-ipn-guinda.png}
        };
        % Líneas decorativas verticales
        \draw[line width=3pt, gold] (-1.5cm,-2.5cm) -- (-1.5cm,-22cm);
        \draw[line width=3pt, ipnBrown] (-0.5cm,-2.5cm) -- (-0.5cm,-22cm);
        % Logo UPIITA (abajo izquierda)
        \node[anchor=south west, xshift=1cm, yshift=2cm] at (current page.south west) {
            \includegraphics[height=2.5cm]{logo-upiita.png}
        };
    \end{tikzpicture}

    % Contenido principal
    \begin{flushright}
    \begin{minipage}{0.85\textwidth}
        \begin{center}
            \vspace{1cm}
            {\LARGE\bfseries INSTITUTO POLITÉCNICO NACIONAL\par}
            \vspace{1.5cm}
            {\Large\bfseries UNIDAD PROFESIONAL INTERDISCIPLINARIA EN\\
            INGENIERÍA Y TECNOLOGÍAS AVANZADAS\par}
            \vspace{1cm}
            {\Large\bfseries U P I I T A\par}
            \vspace{1.5cm}
            
            {\large\itshape ``Arquitectura IoT para el Monitoreo\\
            y Detección de Anomalías en\\
            el Consumo Eléctrico Residencial''\par}
            \vspace{1.5cm}

            {\large Que para obtener el título de\par}
            \vspace{0.5cm}
            {\large\bfseries ``Ingeniero en Telemática''\par}
            \vspace{1.5cm}

            {\large Presenta el alumno:\par}
            \vspace{0.5cm}
            {\large\bfseries Cordero Montes de Oca Luis Alberto\par}
            \vspace{3cm}

            \begin{minipage}{0.45\textwidth}
                \centering
                \rule{6cm}{0.4pt}\\
                {\large Huerta Trujillo Iliac}
            \end{minipage}
            \hfill
            \begin{minipage}{0.45\textwidth}
                \centering
                \rule{6cm}{0.4pt}\\
                {\large Villordo Jimenez Iclia}
            \end{minipage}

        \end{center}
    \end{minipage}
    \end{flushright}

\end{titlepage}
\restoregeometry

% ========================= Páginas preliminares =========================
\pagenumbering{roman}
\setcounter{page}{1}

\chapter*{Agradecimientos}
\addcontentsline{toc}{chapter}{Agradecimientos}
% Texto de agradecimientos.
\chapter*{Resumen}
\addcontentsline{toc}{chapter}{Resumen}
\markboth{Resumen}{Resumen}
% Escribe aquí tu resumen en español.
% Palabras clave: IoT; detección de anomalías; energía; tiempo real.

\chapter*{Abstract}
\addcontentsline{toc}{chapter}{Abstract}
\markboth{Abstract}{Abstract}
\selectlanguage{english}
% Write here your abstract in English.
% Keywords: IoT; anomaly detection; energy; real-time.
\selectlanguage{spanish}

% \printglossary[type=\acronymtype,title={Lista de Acrónimos}] % Si usas glossaries

\tableofcontents
\listoftables
\listoffigures

% ========================= Cuerpo principal =========================
\clearpage
\pagenumbering{arabic}

\chapter{Introducción}
\section{Contexto y motivación}
\section{Planteamiento del problema}
\section{Objetivos}
\subsection{Objetivo general}
\subsection{Objetivos específicos}
\section{Alcance y limitaciones}
\section{Estructura del documento}

\chapter{Estado del Arte}
\section{Arquitecturas IoT}
Las arquitecturas de \gls{iot} proporcionan marcos de referencia para estructurar sistemas distribuidos que abarcan dispositivos físicos, comunicación, procesamiento y aplicaciones. Su correcta selección e implementación es crítica para asegurar requisitos no funcionales como escalabilidad, resiliencia y seguridad, especialmente en escenarios de telemetría energética y detección de consumos anómalos en tiempo real.

\subsection{Modelos de referencia y estandarización}
Existen diversos modelos de referencia que guían el diseño de soluciones IoT, cada uno con énfasis distinto en dominios industriales, interoperabilidad o gobernanza:

\begin{itemize}
  \item \textbf{IoT-A (The IoT Architectural Reference Model)}: Proporciona abstracciones de alto nivel (dominio, información, comunicación, funcional, seguridad) y un metamodelo que promueve interoperabilidad semántica entre objetos y servicios \cite{iota-arm-2013}. Es útil para mantener consistencia conceptual en sistemas heterogéneos.
  \item \textbf{IIRA (Industrial Internet Reference Architecture)} del Industrial Internet Consortium: Estructura el sistema en \emph{viewpoints} (business, usage, functional, implementation) y define dominios funcionales (control, operaciones, información, aplicación) con foco en integración OT/IT y requisitos industriales (seguridad, confiabilidad, disponibilidad) \cite{iic-iira-2019}. Adecuado para plantas industriales y energía.
  \item \textbf{RAMI 4.0} (Reference Architectural Model Industry 4.0): Modelo tridimensional que alinea capas (del activo físico a la comunicación y la integración), jerarquía de la ISA-95 y ciclo de vida del producto \cite{rami4-0}. Útil para trazabilidad, gemelos digitales y conformidad en manufactura/energía.
  \item \textbf{NIST IoT-Reference}: Define funciones nucleares (sensing, communication, data processing, applications, y seguridad transversal) con énfasis en gestión de riesgos y conformidad \cite{nist-sp800-183, nist-iot-core-baseline}.
  \item \textbf{Arquitectura 5C} (Connection, Conversion, Cyber, Cognition, Configuration): Centrada en manufactura inteligente y mantenimiento predictivo, introduce una progresión desde adquisición hasta realimentación autónoma \cite{lee2015cyber}.
\end{itemize}

Para proyectos de monitoreo energético residencial o industrial ligero, IoT-A y NIST facilitan la interoperabilidad y la gestión del riesgo; para entornos industriales con integración de SCADA/PLC, IIRA y RAMI 4.0 aportan patrones robustos y vocabularios alineados a estándares.

\subsection{Capas funcionales y patrón Edge–Fog–Cloud}
Un patrón ampliamente adoptado organiza la solución en capas: \emph{device/edge}, \emph{fog/gateway} y \emph{cloud}, lo que permite distribuir cómputo, almacenamiento y control según latencia y costo \cite{shi2016edge, satyanarayanan2017}.

\begin{enumerate}
  \item \textbf{Capa de Dispositivo/Percepción}: Sensores/actuadores, medición de potencia/energía, muestreo, preprocesamiento ligero (filtro, compresión, extracción de rasgos básicos). Requisitos: bajo consumo, fiabilidad, sincronización temporal.
  \item \textbf{Capa de Borde (Edge/Fog/Gateway)}: Agregación de flujos, normalización, seguridad (terminación TLS), control de calidad de servicio, inferencia de baja latencia (detección temprana de anomalías) y \emph{store-and-forward}. Protocolos típicos: MQTT, CoAP, OPC UA \cite{mqtt-oasis, coap-rfc7252, opcuabook}.
  \item \textbf{Capa de Nube}: Ingesta a gran escala, almacenamiento \emph{time-series}, procesamiento por lotes/streaming, entrenamiento de modelos y visualización. Aquí se prioriza elasticidad, gestión de datos históricos y analítica avanzada.
\end{enumerate}

Este desacoplamiento reduce latencia, delimita dominios de fallo y permite optimizar costos (cómputo intensivo en la nube; decisiones inmediatas en el borde). En detección de consumos anómalos, ejecutar reglas de primer nivel (umbrales, z-scores, modelos ligeros) en el edge mitiga eventos falsos y mejora tiempos de reacción, mientras que la nube refina detecciones con modelos estacionales o de \emph{deep learning}.

\subsection{Planos transversales: seguridad, gestión y observabilidad}
Independientemente del particionado en capas, tres planos transversales sostienen la operación:

\begin{itemize}
  \item \textbf{Seguridad} (\emph{security by design}): identidad del dispositivo (X.509), autenticación/autorización por tópicos MQTT, cifrado extremo a extremo (TLS 1.2+), rotación de credenciales y \emph{secure boot} \cite{nist-iot-core-baseline, mqtt-oasis}. El principio de privilegio mínimo debe reflejarse en ACLs y particionamiento de redes.
  \item \textbf{Gestión del ciclo de vida} (device management): provisión, configuración remota, actualización OTA, inventario y retirada. Estándares como LwM2M/OMA facilitan interoperabilidad \cite{lwm2m}.
  \item \textbf{Observabilidad}: métricas, logs y trazas distribuidas desde edge a cloud; retención diferenciada para series temporales; paneles de control y alertamiento con umbrales adaptativos.
\end{itemize}

\subsection{Mensajería y acoplamiento}
Los buses de mensajería desacoplados permiten elasticidad y resiliencia. \textbf{MQTT} es dominante por su ligereza y soporte de QoS, \textbf{CoAP} para dispositivos extremadamente restringidos, y \textbf{OPC UA} en entornos industriales que requieren modelado semántico y seguridad integrada \cite{mqtt-oasis, coap-rfc7252, opcuabook}. La selección debe considerar:
\begin{enumerate}
  \item \emph{Patrones de comunicación}: \emph{publish/subscribe} vs. \emph{request/response}.
  \item \emph{QoS y latencia}: QoS 1/2 para eventos críticos; retención para últimas lecturas; \emph{backpressure}.
  \item \emph{Esquemas de tópicos y contratos}: nombres jerárquicos (`site/area/device/metric`) y formatos auto-descriptivos (JSON/CBOR) con versión.
\end{enumerate}

\subsection{Datos y analítica para detección de anomalías}
Para consumos anómalos, el diseño de datos es determinante: granularidad temporal, agregaciones por ventana, manejo de estacionalidad, metadatos de dispositivo y normalización. En el borde, modelos ligeros (EWMA, ARIMA básico, bosque aleatorio compacto) pueden filtrar anomalías evidentes; en la nube, enfoques más pesados (LSTM, autoencoders, Prophet) capturan estacionalidad diaria/semanal \cite{laptev2015, malhotra2015, taylor2018prophet}. Es recomendable separar:
\begin{itemize}
  \item \emph{Plano de características} (extracción/ingeniería de variables).
  \item \emph{Plano de inferencia} (serving de modelos con control de versiones).
  \item \emph{Plano de explicación} (importancias, SHAP, reglas).
\end{itemize}

\subsection{Buenas prácticas y decisiones de diseño}
\begin{itemize}
  \item \textbf{Criterio de colocación}: colocar lógica donde minimice latencia y costo y maximice confiabilidad (p. ej., cortes de red → lógica esencial en edge).
  \item \textbf{Desacoplar por contratos}: usar contratos versionados de mensajes y pruebas de compatibilidad.
  \item \textbf{Seguridad desde el inicio}: identidades por dispositivo, TLS obligatorio, ACL por tópico, rotación periódica.
  \item \textbf{Observabilidad end-to-end}: trazabilidad desde sensor hasta alerta; KPIs por etapa.
\end{itemize}

\section{Protocolos y mensajería}
La selección de protocolos de mensajería en \gls{iot} condiciona latencia, consumo energético, confiabilidad y seguridad del sistema. En monitoreo de consumo eléctrico y detección de anomalías, predominan protocolos ligeros orientados a telemetría (\textbf{MQTT}, \textbf{CoAP}) junto con \textbf{HTTP/REST} y \textbf{WebSockets} para integración de aplicaciones, y \textbf{OPC UA} en escenarios industriales con fuerte semántica y requisitos de conformidad \cite{mqtt-oasis,coap-rfc7252,fielding2000rest,opcuabook}.

\subsection{MQTT: telemetría ligera con publish/subscribe}
\textbf{MQTT} es un protocolo de mensajería ligero sobre TCP con patrón \emph{publish/subscribe} y un \emph{broker} central (por ejemplo, Mosquitto, EMQX). Ofrece niveles de calidad de servicio (QoS 0, 1 y 2), retención de mensajes y sesiones persistentes, lo que lo hace idóneo para enlaces inestables y dispositivos con recursos limitados \cite{mqtt-oasis}.
\begin{itemize}
  \item \textit{Topologías y contratos}: la semántica de tópicos suele ser jerárquica (\texttt{site/area/device/metric}) y conviene versionar (\texttt{v1/}) para evitar acoplamiento rígido.
  \item \textit{QoS}: QoS 1/2 reducen pérdida de datos a costa de mayor latencia y tráfico; en energía, métricas críticas (alarmas) suelen usar QoS 1 y lecturas de alta frecuencia, QoS 0.
  \item \textit{Retained \& LWT}: el mensaje retenido entrega el último estado a suscriptores tardíos; el \emph{Last Will and Testament} notifica desconexiones inesperadas para mejorar observabilidad.
  \item \textit{Seguridad}: TLS 1.2+ con autenticación por certificados X.509 y ACL por tópico en el broker; para clientes muy restringidos, credenciales rotadas y políticas de mínimo privilegio \cite{nist-iot-core-baseline}.
\end{itemize}
Variantes como \textbf{MQTT-SN} operan sobre UDP y reducen cabeceras para redes muy restringidas (LPWAN).

\subsection{CoAP: request/response ligero sobre UDP}
\textbf{CoAP} implementa un estilo \emph{RESTful} para dispositivos restringidos sobre UDP, con confirmación opcional (CON/NON), \emph{observe} para notificaciones tipo \emph{server push} y soporte de \textbf{DTLS} para seguridad \cite{coap-rfc7252,dtls-rfc9147}. Su modelo \emph{resource-oriented} y payloads compactos (p. ej., CBOR) son útiles en redes LLN (6LoWPAN).
\begin{itemize}
  \item \textit{Modos confiables}: mensajes confirmables (CON) ofrecen fiabilidad similar a QoS 1; \emph{block-wise transfer} soporta cargas mayores.
  \item \textit{Seguridad}: DTLS 1.3 o OSCORE para cifrado extremo a extremo a nivel de objeto en entornos con proxies.
\end{itemize}

\subsection{HTTP/REST y WebSockets: integración y streaming a aplicaciones}
\textbf{HTTP/REST} sigue dominando la integración con sistemas TI y paneles de control por su ubicuidad y semántica clara \cite{fielding2000rest}. Para telemetría en tiempo casi real a dashboards, \textbf{WebSockets} permiten canal bidireccional y estable reduciendo la sobrecarga de sondeo.
\begin{itemize}
  \item \textit{Patrones}: IoT ingiere vía MQTT/CoAP y expone APIs REST para consulta histórica y gestión; WebSockets para actualizaciones en vivo.
  \item \textit{Seguridad}: TLS obligatorio (HTTPS/WSS), OAuth2/OpenID Connect para autenticación de usuarios y \emph{API keys} o mTLS para \emph{machine-to-machine}.
\end{itemize}

\subsection{OPC UA: semántica y seguridad integradas en OT}
\textbf{OPC UA} combina modelo de información rico, servicios de descubrimiento, suscripción y seguridad integrada (cifrado, firma, políticas de usuario) \cite{opcuabook}. En energía industrial, facilita interoperabilidad con PLC/SCADA y modelado semántico de activos. Para dispositivos muy restringidos, su costo puede ser elevado; es común integrarlo en gateway/edge que publica hacia la nube vía MQTT (puente MQTT–OPC UA).

\subsection{Formatos de payload y serialización}
La carga útil debe equilibrar compacidad, legibilidad y costos de serialización:
\begin{itemize}
  \item \textbf{JSON}: legible y ampliamente soportado; mayor tamaño. Útil para integración rápida.
  \item \textbf{CBOR}: binario compacto y eficiente para dispositivos y CoAP \cite{cbor-rfc8949}.
  \item \textbf{Protocol Buffers}: esquemas versionados y eficientes, ventajosos para pipelines de alto volumen y control de contratos \cite{protobuf}.
\end{itemize}
Se recomienda versionar esquemas, incluir marcas de tiempo normalizadas (UTC, ISO~8601) y metadatos (ID de dispositivo, calidad, unidad) para reproducibilidad y detección de anomalías.

\subsection{Patrones de diseño de mensajería}
\begin{itemize}
  \item \textbf{Topic design}: jerarquías por ubicación/activo/métrica, evitando comodines en producción para limitar exposición y costos.
  \item \textbf{Backpressure}: uso de colas/búferes en edge (\emph{store-and-forward}), límites de \emph{in-flight} y gestión de reconexiones exponenciales.
  \item \textbf{Idempotencia y orden}: claves de deduplicación y \emph{sequencing} para series temporales; en MQTT, no hay garantía de orden global con múltiples publicadores.
  \item \textbf{QoS-aware routing}: enrutar mensajes críticos por canales con QoS superior; agrupar telemetría de baja prioridad en lotes para eficiencia.
\end{itemize}

\subsection{Seguridad de transporte y extremos}
\begin{itemize}
  \item \textbf{TLS/DTLS}: cifrado en tránsito, PFS y suites modernas (AES-GCM/ChaCha20-Poly1305) \cite{dtls-rfc9147}.
  \item \textbf{mTLS}: autenticación mutua para dispositivos y gateways; gestión de PKI con rotación y revocación.
  \item \textbf{Autorización por tópico/recurso}: ACL por patrón de tópico (MQTT) o ruta (CoAP/HTTP); principio de mínimo privilegio.
  \item \textbf{E2E en entornos con proxy}: OSCORE (CoAP) y cifrado a nivel de payload cuando el proxy debe operar solo con metadatos.
\end{itemize}

\subsection{Criterios de selección para consumo energético y anomalías}
Para detección de consumos anómalos:
\begin{itemize}
  \item \textbf{MQTT} con QoS 1 para eventos/alarmas y QoS 0 para telemetría de alta frecuencia, retenidos para estados; TLS y ACL estrictas en broker.
  \item \textbf{CoAP} con confirmables (CON) y CBOR cuando los dispositivos sean muy restringidos o operen sobre 6LoWPAN/LPWAN.
  \item \textbf{HTTP/REST} para APIs de consulta histórica y administración; \textbf{WebSockets} para dashboards en vivo.
  \item \textbf{Puentes OPC UA–MQTT} cuando se integra OT/SCADA con analítica en la nube.
\end{itemize}
La decisión final debe balancear consumo de ancho de banda, latencia requerida para alertas, capacidad del dispositivo y requisitos de seguridad y conformidad.
\section{Plataformas y servicios}
Las plataformas para \gls{iot} integran la cadena de valor desde la ingesta de telemetría hasta la analítica y visualización. En escenarios de consumo energético y detección de anomalías, los componentes clave incluyen: \emph{brokers} de mensajería, \emph{ingestors}, almacenamiento de series temporales, procesamiento de flujos y lotes, orquestación y tableros. La selección debe balancear latencia, costo, elasticidad, gobernanza y seguridad.

\subsection{Brokers de mensajería y gateways}
Los \textbf{brokers MQTT} (Eclipse Mosquitto, EMQX, HiveMQ) son la columna vertebral de la telemetría \cite{mqtt-oasis}. Ofrecen \emph{topic ACLs}, persistencia, retención, \emph{bridges} y plugins para extender autenticación/autorización y observabilidad. \textbf{CoAP} suele desplegarse con \emph{proxies} y \emph{resource directories} para descubrimiento \cite{coap-rfc7252}. En entornos industriales, \textbf{OPC UA} se integra en \emph{gateways} que traducen a MQTT para la nube \cite{opcuabook}.
\begin{itemize}
  \item \textit{Criterios}: soporte de TLS/mTLS, control fino de ACL por tópico, cuotas, gestión de sesiones, \emph{bridging} multi-\emph{tenant}, métricas nativas.
  \item \textit{Operación}: alta disponibilidad con clúster activo-activo y \emph{retained stores} replicados; backpressure y límites \emph{in-flight}.
\end{itemize}

\subsection{Ingesta y canalización}
La ingesta desacopla dispositivos de servicios de almacenamiento/analítica. Patrones comunes:
\begin{itemize}
  \item \textbf{Puentes MQTT→TSDB/Stream}: conectores nativos o \emph{functions} del broker para enrutar a bases de series de tiempo o buses de streaming.
  \item \textbf{Buses de datos} (Kafka/Pulsar): proporcionan \emph{durability}, particionado y \emph{replay} para flujos de alta tasa \cite{kafka, pulsar}. Útiles cuando múltiples consumidores (detección online, archivado, aprendizaje) comparten el mismo \emph{topic}.
  \item \textbf{ETL/ELT}: normalización, validación de esquemas (JSON Schema/Protobuf), enriquecimiento (metadatos de dispositivo) y control de versiones de contratos.
\end{itemize}

\subsection{Almacenamiento de series temporales (TSDB)}
Las \textbf{TSDB} optimizan escrituras en ráfaga, compresión y \emph{downsampling} con retención por políticas. Alternativas populares incluyen \textbf{InfluxDB}, \textbf{TimescaleDB} y \textbf{Prometheus} (sobre todo para métricas operativas) \cite{influxdb, timescale, prometheus}. Para consumo energético:
\begin{itemize}
  \item \textit{Requisitos}: escrituras sostenidas, agregaciones por ventana, alineación temporal, etiquetado por \texttt{device/site/metric}, y consultas por rango.
  \item \textit{Esquemas}: diseño basado en \emph{measurement} + \emph{tags} (InfluxDB) o tablas particionadas por tiempo (TimescaleDB); compresión \emph{delta-of-delta} y \emph{gorilla} para eficiencia.
  \item \textit{Retención}: políticas multi-\emph{tier} (crudo por días, agregados por meses) para optimizar costos.
\end{itemize}

\begin{table}[H]
\centering
\caption{Comparativa de TSDBs para telemetría IoT energética}
\label{tab:tsdb-comparativa}
\renewcommand{\arraystretch}{1.2}
\begin{tabular}{@{}p{3.0cm} p{3.2cm} p{3.2cm} p{3.2cm}@{}}
\toprule
\textbf{Criterio} & \textbf{InfluxDB} & \textbf{TimescaleDB} & \textbf{Prometheus} \\
\midrule
Modelo de datos & Measurements + tags/fields; series identificadas por tags & PostgreSQL + extensiones; hypertables/particiones por tiempo & Series etiquetadas (metrics + labels); enfoque en métricas operativas \\
Ingesta & Alta tasa nativa; Line Protocol; conectores MQTT/Kafka & Ingesta vía SQL/FDW/ingestores; buena integración con ecosistema SQL & Pull (scrape) y push vía gateway; formato de texto \\
Consultas & Flux/SQL (según edición); agregaciones y downsampling & SQL estándar + funciones de time-series; facilidad de joins & PromQL; consultas operativas para monitoreo/alertas \\
Compresión & Delta-of-delta, Gorilla, TSM/TSI & Compresión de PostgreSQL + compresión por chunks & Compresión específica por bloques en TSDB \\
Retención & Políticas por bucket; downsampling automático & Políticas por partición/retención; administración vía SQL & Retención por tiempo/espacio; pensado para métricas \\
Escalabilidad & OSS single-node; Enterprise/Cloud con clustering & Escala vertical + particionado; replicación nativa de Postgres & Federación y sharding vía proyectos complementarios \\
Casos de uso típicos & Telemetría IoT, sensores, analítica temporal general & IoT + analítica relacional; integración con datos de negocio & Observabilidad (métricas app/infra), alertas, paneles \\
Licencia & OSS + ofertas comerciales (Cloud/Enterprise) & OSS (Timescale License para algunas features) & OSS (Apache 2.0) \\
Ecosistema & Conectores a Grafana, Telegraf, Kapacitor & Ecosistema Postgres, extensiones, Grafana & Stack de observabilidad, Grafana, Alertmanager \\
Adecuación a consumo energético & Muy buena para series densas y downsampling & Excelente si necesitas joins con datos de negocio y SQL & Útil para métricas operativas; menos óptimo para series crudas densas \\
\bottomrule
\end{tabular}
\end{table}

\subsection{Procesamiento de flujos y lotes}
La detección de anomalías requiere tanto \textbf{stream processing} (detección temprana) como \textbf{batch} (modelado y recalibración):
\begin{itemize}
  \item \textbf{Stream}: Apache Flink, Spark Structured Streaming, Kafka Streams permiten ventanas deslizantes/tumbling, \emph{watermarks} y estado gestionado para reglas y modelos online \cite{flink, spark-ss}.
  \item \textbf{Batch}: Spark/Beam para entrenamiento y análisis histórico; orquestado por Airflow/Luigi con \emph{feature stores}.
  \item \textit{Model serving}: \emph{microservicios} con REST/gRPC o \emph{serverless} para inferencia; control de versiones de modelos y \emph{shadow deployments}.
\end{itemize}

\subsection{Orquestación y contenedores}
La estandarización de despliegues facilita portabilidad del borde a la nube:
\begin{itemize}
  \item \textbf{Contenedores} (Docker/OCI) y \textbf{Kubernetes} para elasticidad y gestión declarativa \cite{kubernetes}. En el borde, \textbf{K3s} y \textbf{Azure IoT Edge}/\textbf{AWS IoT Greengrass} facilitan módulos, \emph{twin state} y despliegues OTA \cite{azure-iot-edge, aws-greengrass}.
  \item \textbf{IaC}: Terraform/Ansible para reproducibilidad; \emph{secrets management} (Vault/KMS) y políticas de seguridad.
\end{itemize}

\subsection{Dashboards, APIs y alertamiento}
La plataforma debe exponer:
\begin{itemize}
  \item \textbf{APIs} REST/GraphQL para consulta histórica y administración; \textbf{WebSockets} para actualizaciones en vivo.
  \item \textbf{Dashboards} (Grafana, Kibana, superset) con paneles por dispositivo, métricas agregadas y eventos \cite{grafana}.
  \item \textbf{Alertas}: umbrales dinámicos, detecciones por modelo y correlación de eventos; integración con email, webhooks y mensajería.
\end{itemize}

\subsection{Seguridad, gobernanza y cumplimiento}
Las plataformas deben incorporar \emph{security-by-default} y gobernanza de datos:
\begin{itemize}
  \item \textbf{Identidad y acceso}: mTLS para dispositivos/gateways; IAM centralizado; ACL por tópico/recurso; rotación de credenciales y revocación.
  \item \textbf{Data governance}: clasificación de datos, \emph{data lineage}, retención y anonimización donde aplique.
  \item \textbf{Cumplimiento}: lineamientos NIST para capacidad base de ciberseguridad en IoT \cite{nist-iot-core-baseline}.
\end{itemize}

\subsection{Criterios de selección para IoT energético}
\begin{itemize}
  \item \textbf{Broker} con TLS/mTLS, ACL por tópico, sesiones persistentes y métricas exportables.
  \item \textbf{TSDB} con retención por niveles y consultas eficientes por \emph{tags}.
  \item \textbf{Stream processor} con ventanas y estado para reglas de primer nivel; \textbf{batch} para recalibración de modelos.
  \item \textbf{Dashboards} con soportes de anotaciones y correlaciones; APIs bien definidas para integración externa.
\end{itemize}



\section{Trabajos relacionados}

\chapter{Marco Teórico}
\section{Fundamentos de medición y energía}
\section{Protocolos IoT y seguridad}
\section{Modelos y técnicas de detección de anomalías}

\chapter{Análisis del Sistema}
\section{Requerimientos funcionales}
\section{Requerimientos no funcionales}
\section{Casos de uso y actores}
\section{Riesgos y supuestos}

\chapter{Diseño del Sistema}
\section{Arquitectura propuesta}
\section{Diseño de datos y tópicos}
\section{Diseño de servicios e interfaces}
\section{Diagramas de componentes y secuencia}

\chapter{Implementación del Sistema}
\section{Entorno y herramientas}
\section{Módulos implementados}
\section{Configuración y despliegue}
\section{Ejemplos de código}
% \begin{mdframed}
% \begin{verbatim}
%   Fragmentos de configuración o comandos
% \end{verbatim}
% \end{mdframed}

\chapter{Pruebas y Resultados}
\section{Plan y metodología de pruebas}
\section{Pruebas funcionales e integración}
\section{Rendimiento y escalabilidad}
\section{Resultados y discusión}

\chapter{Conclusiones y Trabajo Futuro}
\section{Conclusiones}
\section{Limitaciones}
\section{Líneas de trabajo futuro}

% ========================= Apéndices =========================
\appendix
\chapter{Guía de despliegue}
\chapter{Datasets e instrumentación}
\chapter{Documentación adicional}

% ========================= Referencias =========================
\clearpage
\chapter*{Referencias}
\addcontentsline{toc}{chapter}{Referencias}
\printbibliography[heading=none]

\end{document}