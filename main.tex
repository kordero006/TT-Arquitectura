% !TeX program = pdflatex
\documentclass[12pt,openright,oneside]{report}

% ========================= Codificación y tipografía (pdfLaTeX) =========================
\usepackage[utf8]{inputenc}
\usepackage[T1]{fontenc}
\usepackage{lmodern}

% ========================= Idiomas =========================
\usepackage[spanish,english]{babel}
\selectlanguage{spanish}

% ========================= Geometría, espaciado y tablas =========================
\usepackage{geometry}
\geometry{a4paper, margin=2.5cm}
\usepackage{setspace}
% \onehalfspacing % Activa si tu normativa lo exige
\usepackage{booktabs}
\usepackage{longtable}
\usepackage{multirow}
\usepackage{array}

% ========================= Gráficos y dibujo =========================
\usepackage{graphicx}
\graphicspath{{./}{figuras/}} % Ajusta si tus logos están en otra ruta
\usepackage{float}
\usepackage{subcaption}
\usepackage{tikz}
\usepackage{pgfplots}
\usepgfplotslibrary{colorbrewer}
\pgfplotsset{width=8cm,compat=1.18}

% ========================= Matemáticas y entornos =========================
\usepackage{amsmath, amssymb, amsthm, bm}
\usepackage{empheq}
\usepackage{mdframed}

% ========================= Colores =========================
\usepackage{xcolor}
\definecolor{ocre}{RGB}{243,102,25}
\definecolor{mygray}{RGB}{243,243,244}
\definecolor{deepGreen}{RGB}{26,111,0}
\definecolor{shallowGreen}{RGB}{235,255,255}
\definecolor{deepBlue}{RGB}{61,124,222}
\definecolor{shallowBlue}{RGB}{235,249,255}
\definecolor{ipnBrown}{RGB}{123,48,42}
\definecolor{gold}{RGB}{212,175,55}

% Caja sencilla
\newcommand\orangebox[1]{\fcolorbox{ocre}{mygray}{\hspace{1em}#1\hspace{1em}}}

% Entornos tipo teorema (nombres en inglés como en tu versión)
\newtheoremstyle{mytheoremstyle}{3pt}{3pt}{\normalfont}{0cm}{\rmfamily\bfseries}{}{1em}{{\color{black}\thmname{#1}~\thmnumber{#2}}\thmnote{\,--\,#3}}
\newtheoremstyle{myproblemstyle}{3pt}{3pt}{\normalfont}{0cm}{\rmfamily\bfseries}{}{1em}{{\color{black}\thmname{#1}~\thmnumber{#2}}\thmnote{\,--\,#3}}
\theoremstyle{mytheoremstyle}
\newmdtheoremenv[linewidth=1pt,backgroundcolor=shallowGreen,linecolor=deepGreen,leftmargin=0pt,innerleftmargin=20pt,innerrightmargin=20pt]{theorem}{Theorem}[section]
\theoremstyle{mytheoremstyle}
\newmdtheoremenv[linewidth=1pt,backgroundcolor=shallowBlue,linecolor=deepBlue,leftmargin=0pt,innerleftmargin=20pt,innerrightmargin=20pt]{definition}{Definition}[section]
\theoremstyle{myproblemstyle}
\newmdtheoremenv[linecolor=black,leftmargin=0pt,innerleftmargin=10pt,innerrightmargin=10pt]{problem}{Problem}[section]

% ========================= Encabezados y pies =========================
\usepackage{fancyhdr}
\pagestyle{fancy}
\fancyhf{}
\fancyhead[LE,RO]{\thepage}
\fancyhead[LO]{\nouppercase{\leftmark}}
\fancyhead[RE]{\nouppercase{\rightmark}}

% ========================= Hipervínculos y metadatos =========================
\usepackage[hidelinks]{hyperref}
\hypersetup{
  pdftitle={Prueba de Proyecto},
  pdfauthor={Luis Alberto Cordero Montes de Oca},
  pdfsubject={Arquitectura IoT para la detección de consumos anómalos},
  pdfkeywords={IoT, energía, anomalías, tiempo real}
}

% ========================= Glosarios/Acrónimos (opcional) =========================
% Si tu guía lo permite, puedes descomentar para lista de acrónimos.
% \usepackage[acronym]{glossaries}
% \makeglossaries
% \newacronym{iot}{IoT}{Internet de las Cosas}
% \newacronym{mqtt}{MQTT}{Message Queuing Telemetry Transport}

% ========================= Bibliografía (opcional, recomendado) =========================
% Descomenta si vas a usar .bib
% \usepackage[backend=biber,style=ieee,sorting=nty]{biblatex}
% \addbibresource{referencias.bib}

% ========================= Título y autor (para metadatos; no se usa portada automática) =========================
\title{Prueba de Proyecto}
\author{Luis Alberto Cordero Montes de Oca}

\begin{document}

% ========================= Portada (MANTENIDA TAL CUAL) =========================
\begin{titlepage}
    \thispagestyle{empty}
    \newgeometry{left=2.5cm,right=2.5cm,top=2.5cm,bottom=2.5cm}

    \begin{tikzpicture}[remember picture, overlay]
        % Logo IPN (arriba izquierda)
        \node[anchor=north west, xshift=1cm, yshift=-2cm] at (current page.north west) {
            \includegraphics[height=3cm]{logo-ipn-guinda.png}
        };
        % Líneas decorativas verticales
        \draw[line width=3pt, gold] (-1.5cm,-2.5cm) -- (-1.5cm,-22cm);
        \draw[line width=3pt, ipnBrown] (-0.5cm,-2.5cm) -- (-0.5cm,-22cm);
        % Logo UPIITA (abajo izquierda)
        \node[anchor=south west, xshift=1cm, yshift=2cm] at (current page.south west) {
            \includegraphics[height=2.5cm]{logo-upiita.png}
        };
    \end{tikzpicture}

    % Contenido principal
    \begin{flushright}
    \begin{minipage}{0.85\textwidth}
        \begin{center}
            \vspace{1cm}
            {\LARGE\bfseries INSTITUTO POLITÉCNICO NACIONAL\par}
            \vspace{1.5cm}
            {\Large\bfseries UNIDAD PROFESIONAL INTERDISCIPLINARIA EN\\
            INGENIERÍA Y TECNOLOGÍAS AVANZADAS\par}
            \vspace{1cm}
            {\Large\bfseries U P I I T A\par}
            \vspace{1.5cm}

            {\large\itshape ``Arquitectura IoT para la Detección de\\
            Consumos Anómalos y Seguimiento en\\
            Tiempo Real de Dispositivos Eléctricos''\par}
            \vspace{1.5cm}

            {\large Que para obtener el título de\par}
            \vspace{0.5cm}
            {\large\bfseries ``Ingeniero en Telemática''\par}
            \vspace{1.5cm}

            {\large Presenta el alumno:\par}
            \vspace{0.5cm}
            {\large\bfseries Cordero Montes de Oca Luis Alberto\par}
            \vspace{3cm}

            \begin{minipage}{0.45\textwidth}
                \centering
                \rule{6cm}{0.4pt}\\
                {\large Huerta Trujillo Iliac}
            \end{minipage}
            \hfill
            \begin{minipage}{0.45\textwidth}
                \centering
                \rule{6cm}{0.4pt}\\
                {\large Villordo Jimenez Iclia}
            \end{minipage}

        \end{center}
    \end{minipage}
    \end{flushright}

\end{titlepage}
\restoregeometry

% ========================= Páginas preliminares =========================
\pagenumbering{roman}
\setcounter{page}{1}

\chapter*{Agradecimientos}
\addcontentsline{toc}{chapter}{Agradecimientos}
% Texto de agradecimientos.
\chapter*{Resumen}
\addcontentsline{toc}{chapter}{Resumen}
\markboth{Resumen}{Resumen}
% Escribe aquí tu resumen en español.
% Palabras clave: IoT; detección de anomalías; energía; tiempo real.

\chapter*{Abstract}
\addcontentsline{toc}{chapter}{Abstract}
\markboth{Abstract}{Abstract}
\selectlanguage{english}
% Write here your abstract in English.
% Keywords: IoT; anomaly detection; energy; real-time.
\selectlanguage{spanish}

% \printglossary[type=\acronymtype,title={Lista de Acrónimos}] % Si usas glossaries

\tableofcontents
\listoftables
\listoffigures

% ========================= Cuerpo principal =========================
\clearpage
\pagenumbering{arabic}

\chapter{Introducción}
\section{Contexto y motivación}
\section{Planteamiento del problema}
\section{Objetivos}
\subsection{Objetivo general}
\subsection{Objetivos específicos}
\section{Alcance y limitaciones}
\section{Estructura del documento}

\chapter{Estado del Arte}
\section{Arquitecturas IoT}
\section{Protocolos y mensajería}
\section{Plataformas y servicios}
\section{Trabajos relacionados}

\chapter{Marco Teórico}
\section{Fundamentos de medición y energía}
\section{Protocolos IoT y seguridad}
\section{Modelos y técnicas de detección de anomalías}

\chapter{Análisis del Sistema}
\section{Requerimientos funcionales}
\section{Requerimientos no funcionales}
\section{Casos de uso y actores}
\section{Riesgos y supuestos}

\chapter{Diseño del Sistema}
\section{Arquitectura propuesta}
\section{Diseño de datos y tópicos}
\section{Diseño de servicios e interfaces}
\section{Diagramas de componentes y secuencia}

\chapter{Implementación del Sistema}
\section{Entorno y herramientas}
\section{Módulos implementados}
\section{Configuración y despliegue}
\section{Ejemplos de código}
% \begin{mdframed}
% \begin{verbatim}
%   Fragmentos de configuración o comandos
% \end{verbatim}
% \end{mdframed}

\chapter{Pruebas y Resultados}
\section{Plan y metodología de pruebas}
\section{Pruebas funcionales e integración}
\section{Rendimiento y escalabilidad}
\section{Resultados y discusión}

\chapter{Conclusiones y Trabajo Futuro}
\section{Conclusiones}
\section{Limitaciones}
\section{Líneas de trabajo futuro}

% ========================= Apéndices =========================
\appendix
\chapter{Guía de despliegue}
\chapter{Datasets e instrumentación}
\chapter{Documentación adicional}

% ========================= Referencias =========================
\clearpage
\chapter*{Referencias}
\addcontentsline{toc}{chapter}{Referencias}
% Si usas biblatex:
% \printbibliography

% Si NO usas biblatex, puedes listar manualmente:
\begin{thebibliography}{99}
% \bibitem{ref1} Autor, Título, Editorial, Año.
\end{thebibliography}

\end{document}